\documentclass[paper=a4, fontsize=11pt]{article}
\usepackage[utf8]{inputenc}
\usepackage[english,magyar]{babel}
\usepackage{amsmath}
\usepackage{graphicx} 
\usepackage{float}
\usepackage{rotating}
\usepackage{latexsym}
\usepackage{blindtext}
\addtolength{\oddsidemargin}{-.875in}
\addtolength{\evensidemargin}{-.875in}
\addtolength{\textwidth}{1.95in}
\addtolength{\topmargin}{-.9in}
\addtolength{\textheight}{1.6in}


\title{\scshape\Huge Sejtautomaták }
\date{\scshape\Large 2018.05.5.}
\author{\scshape\huge Nagy Péter\\\scshape\huge m07ilf}









\begin{document}
\maketitle
\newpage

 
\tableofcontents
\newpage




\section{Conway élet játék}
A sejtautomaták egyik legismertebbike a John Conway által kifejlesztett életjáték. Ebben a modellben a sejtjeink egy sakktábla szerű terepen helyezkednek el, ahol minden sejtnek nyolc darab szomszédja van. A sejtek két féle álapotban lehetnek, vagy élő vagy halott állapotban. A rendszer diszkrét lépésekben fejlödik és a sejtek müködése a következő:

\begin{itemize}
\item Ha a sejtnek n élő szomszédja van akkor a sejt állapota nem változik
\item Ha n+1 szomszédja van akkor a sejt élő lesz, függetlenül a jelenlegi állapotától
\item Minden más esetben a sejt elpusztul
\end{itemize}


Az életjáték sok összetett rendszer növekedését, csökkenését vagy mozgását tudja szimulálni. 
A szimuláció Turing-teljes vagyis bármit amit kilehet algoritmusokkal számolni azt képes kiszámolni.Conway egyik sejtése az volt, hogy a növekedésnek van egy felső korláta. 1970-ben ötven dolláros jutalmat kínált azért, hogy ezt valakí igazolja vagy cáfolja.













\subsection{Nyílt peremfeltétel}
Első esetben nézzük meg milyen lesz a nyílt határokkal a szimulációt. 
\\
\\
\\
n=1 eset:
\begin{tabular}{cccc}
1&1&1&0\\
0&0&1&0\\
1&1&0&0\\
0&0&1&1
\end{tabular}
\quad
\begin{tabular}{cccc}
1&0&1&1\\
0&0&0&1\\
0&0&0&0\\
1&0&1&1
\end{tabular}
\quad
\begin{tabular}{cccc}
0&1&1&1\\
1&1&0&1\\
0&1&0&0\\
0&1&1&1
\end{tabular}
\quad
\begin{tabular}{cccc}
0&0&0&1\\
1&0&0&1\\
0&0&0&0\\
1&1&0&1
\end{tabular}
\\
\\
\\
n=2 eset:
\begin{tabular}{cccc}
1&1&1&0\\
0&0&1&0\\
1&1&0&0\\
0&0&1&1
\end{tabular}
\quad
\begin{tabular}{cccc}
0&1&1&0\\
0&0&1&0\\
0&1&0&1\\
0&1&1&0
\end{tabular}
\quad
\begin{tabular}{cccc}
0&1&1&0\\
0&0&0&1\\
0&1&0&1\\
0&1&1&0
\end{tabular}
\quad
\begin{tabular}{cccc}
0&0&1&0\\
0&1&0&1\\
0&1&0&1\\
0&1&1&0
\end{tabular}
\\
\\
\\
n=3 eset:
\begin{tabular}{cccc}
1&1&1&0\\
0&0&1&0\\
1&1&0&0\\
0&0&1&1
\end{tabular}
\quad
\begin{tabular}{cccc}
0&1&0&0\\
0&0&1&0\\
0&1&1&0\\
0&0&0&0
\end{tabular}
\quad
\begin{tabular}{cccc}
0&0&0&0\\
0&1&1&0\\
0&0&0&0\\
0&0&0&0
\end{tabular}
\quad
\begin{tabular}{cccc}
0&0&1&0\\
0&0&0&0\\
0&0&0&0\\
0&0&0&0
\end{tabular}
\\
\\
\\
n=4 eset:
\begin{tabular}{cccc}
1&1&1&0\\
0&0&1&0\\
1&1&0&0\\
0&0&1&1
\end{tabular}
\quad
\begin{tabular}{cccc}
0&0&0&0\\
1&0&0&0\\
0&0&0&0\\
0&0&0&0
\end{tabular}
\quad
\begin{tabular}{cccc}
0&0&0&0\\
0&0&0&0\\
0&0&0&0\\
0&0&0&0
\end{tabular}
\quad
\begin{tabular}{cccc}
0&0&0&0\\
0&0&0&0\\
0&0&0&0\\
0&0&0&0
\end{tabular}
\\
\\
\\
n=4 eset:
\begin{tabular}{cccc}
1&1&1&0\\
0&0&1&0\\
1&1&0&0\\
0&0&1&1
\end{tabular}
\quad
\begin{tabular}{cccc}
0&0&0&0\\
0&1&0&0\\
0&0&0&0\\
0&0&0&0
\end{tabular}
\quad
\begin{tabular}{cccc}
0&0&0&0\\
0&0&0&0\\
0&0&0&0\\
0&0&0&0
\end{tabular}
\quad
\begin{tabular}{cccc}
0&0&0&0\\
0&0&0&0\\
0&0&0&0\\
0&0&0&0
\end{tabular}
\\
\\
\\
n=5 eset:
\begin{tabular}{cccc}
1&1&1&0\\
0&0&1&0\\
1&1&0&0\\
0&0&1&1
\end{tabular}
\quad
\begin{tabular}{cccc}
0&0&0&0\\
0&0&0&0\\
0&0&0&0\\
0&0&0&0
\end{tabular}
\quad
\begin{tabular}{cccc}
0&0&0&0\\
0&0&0&0\\
0&0&0&0\\
0&0&0&0
\end{tabular}
\quad
\begin{tabular}{cccc}
0&0&0&0\\
0&0&0&0\\
0&0&0&0\\
0&0&0&0
\end{tabular}









\subsection{Élő határ}
Ebben az esetben vizsgáljuk az életjátékot különböző n értékekre úgy, hogy a határon mindenhol élő sejteket feltételezünk.
\\
\\
\\
n=1 eset:
\begin{tabular}{cccc}
0&1&0&1\\
0&0&0&0\\
0&1&1&0\\
1&0&0&1
\end{tabular}
\quad
\begin{tabular}{cccc}
0&0&0&0\\
0&0&0&0\\
0&1&1&0\\
0&0&0&0
\end{tabular}
\quad
\begin{tabular}{cccc}
0&0&0&0\\
0&1&1&0\\
0&1&1&0\\
0&0&0&0
\end{tabular}
\quad
\begin{tabular}{cccc}
0&0&0&0\\
0&0&0&0\\
0&0&0&0\\
0&0&0&0
\end{tabular}
\newline
\\
\\
\\
n=2 eset:
\begin{tabular}{cccc}
0&1&0&1\\
0&0&0&0\\
0&1&1&0\\
1&0&0&1
\end{tabular}
\quad
\begin{tabular}{cccc}
0&1&0&0\\
0&1&0&0\\
0&1&1&0\\
0&0&0&0
\end{tabular}
\quad
\begin{tabular}{cccc}
0&0&0&0\\
0&1&0&0\\
0&1&1&0\\
0&0&0&0
\end{tabular}
\quad
\begin{tabular}{cccc}
0&0&0&0\\
0&0&0&0\\
0&0&0&0\\
0&0&0&0
\end{tabular}
\\
\\
\\
n=3 eset:
\begin{tabular}{cccc}
0&1&0&1\\
0&0&0&0\\
0&1&1&0\\
1&0&0&1
\end{tabular}
\quad
\begin{tabular}{cccc}
0&1&0&0\\
0&0&1&0\\
1&0&0&0\\
0&0&0&0
\end{tabular}
\quad
\begin{tabular}{cccc}
0&1&0&0\\
0&0&0&1\\
1&0&0&1\\
0&1&0&0
\end{tabular}
\quad
\begin{tabular}{cccc}
0&1&0&0\\
0&0&0&1\\
1&0&0&1\\
0&1&0&0
\end{tabular}
\\
\\
\\
n=4 eset:
\begin{tabular}{cccc}
0&1&0&1\\
0&0&0&0\\
0&1&1&0\\
1&0&0&1
\end{tabular}
\quad
\begin{tabular}{cccc}
0&0&1&1\\
1&0&0&1\\
0&0&0&1\\
0&0&0&0
\end{tabular}
\quad
\begin{tabular}{cccc}
0&1&1&0\\
0&0&0&0\\
1&0&0&1\\
1&0&0&0
\end{tabular}
\quad
\begin{tabular}{cccc}
0&1&1&0\\
1&0&0&1\\
1&0&0&0\\
0&1&0&0
\end{tabular}
\\
\\
\\
n=5 eset:
\begin{tabular}{cccc}
0&1&0&1\\
0&0&0&0\\
0&1&1&0\\
1&0&0&1
\end{tabular}
\quad
\begin{tabular}{cccc}
1&0&0&1\\
0&0&0&0\\
0&0&0&0\\
1&1&1&1
\end{tabular}
\quad
\begin{tabular}{cccc}
1&0&0&1\\
0&0&0&0\\
0&0&0&0\\
1&1&1&1
\end{tabular}
\quad
\begin{tabular}{cccc}
1&0&0&1\\
0&0&0&0\\
0&0&0&0\\
1&1&1&1
\end{tabular}
\\
\\
\\
n=6 eset:
\begin{tabular}{cccc}
0&1&0&1\\
0&0&0&0\\
0&1&1&0\\
1&0&0&1
\end{tabular}
\quad
\begin{tabular}{cccc}
0&0&0&0\\
0&0&0&0\\
0&0&0&0\\
1&0&0&1
\end{tabular}
\quad
\begin{tabular}{cccc}
0&0&0&0\\
0&0&0&0\\
0&0&0&0\\
0&0&0&0
\end{tabular}
\quad
\begin{tabular}{cccc}
0&0&0&0\\
0&0&0&0\\
0&0&0&0\\
0&0&0&0
\end{tabular}
\\
\\
\\
n=7 eset:
\begin{tabular}{cccc}
0&1&0&1\\
0&0&0&0\\
0&1&1&0\\
1&0&0&1
\end{tabular}
\quad
\begin{tabular}{cccc}
0&0&0&0\\
0&0&0&0\\
0&0&0&0\\
0&0&0&0
\end{tabular}
\quad
\begin{tabular}{cccc}
0&0&0&0\\
0&0&0&0\\
0&0&0&0\\
0&0&0&0
\end{tabular}
\quad
\begin{tabular}{cccc}
0&0&0&0\\
0&0&0&0\\
0&0&0&0\\
0&0&0&0
\end{tabular}



\subsection{Periodikus határ}








\subsection{Véletlen eloszlású hatér}












\section{Két dimenziós homokdomb modell}


\includegraphics[width=\textwidth]{homok}





\newpage
\section{Függelék}
\subsection{Conway élet játék}



\begin{thebibliography}{}

 
\bibitem{jegyzet} 
Jegyzet
\\\texttt{$https://stegerjozsef.web.elte.hu/teaching/szamszim/popdin.pdf$}
 

\end{thebibliography}
















































\end{document}

